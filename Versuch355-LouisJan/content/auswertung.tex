\section{Auswertung}
\label{sec:Auswertung}

Folgende Bauteile sind in der verwendeten Schaltung 1 vorhanden und haben folgende Größen:
$$
  L = 32.351\,mH, \quad
  C = 0,801\,nF, \quad
  C_{Sp} = 0,037\,nF, \quad
  R = 48 \Omega
$$
Dabei waren für den Koppelkondensator folgende Größen einstellbar:
\begin{table}[!h]
    \centering
    \caption{Werte für $C_k$}
    \begin{tabular}{c}
      \toprule
      $C_k/\,nF$ \\
      \midrule
      9,99 \\
      8,00 \\
      6,47 \\
      5,02 \\
      4,00 \\
      3,00 \\
      2,03 \\
      1,01 \\
      \bottomrule
    \end{tabular}
\end{table}

Dabei hat $C_k$ einen Fehler von 3\text{\%}.

\subsection{Bestimmung der Resonanzfrequenz}

Um die Resonanzfrequenz $f_+$ der beiden Kreise anzugleichen, muss zunächst die Resonanzfrequenz des
ersten Kreises mit Hilfe der Lissajous-Figuren bestimmt werden. Diese ist erreicht, wenn
die Lissajous-Figur eine Gerade ergibt, die Phase also gleich null ist.
Um das Fenster zu verkleinern, in dem nach der Resonanzfrequenz gesucht werden muss, wird mit der theoretische Wert
im Vorfeld berechnet:
$$
  f_{th} = \frac{1}{2 \pi \sqrt{LC}} = 31,255\,kHz
$$
Für die Resonanzfrequenz ergab sich:
$$
  f_{exp} = 30,3\,kHz
$$
Damit folgt eine Abweichung vom theoretischen zum expermintellen Wert von 1,03\text{\%}.
\subsection{Verhältnis zwischen Schwingungs- und Schwebungsfrequenz}
\label{sec:vss}

Das Verhältnis zwischen Schwingungs- und Schwebungsfrequenz($\eta$) wird bestimmt, indem die Anzahl der Schwingungsbäuche durch die Gesamtzahl der in ihnen
vorhandenen Maxima geteilt wird. Die genaue Berechnung erfolgt über:

\begin{align}
  \eta_{th} = \frac{f_- + f_+}{2(f_- - f_+)}
\end{align}

Dabei gilt für $f_+$:

\begin{align}
  f_+ = \frac{1}{2\pi\sqrt{L(C+C_{Sp})}}\,.
\end{align}

Und für $f_-$:

\begin{align}
  f_- = \frac{1}{2\pi\sqrt{L(\frac{1}{C}+\frac{2}{C_k})^{-1} + LC_{Sp}}}\,.
\end{align}

Daraus folgt mit den experimentell gemessenen Wert($\eta_{exp}$) und den theoretisch bestimmten Werten($f_-\text{und}\,f_+$), inklusive der
Abweichung von $\eta_{exp}$ zu $\eta_{th}$, folgende Tabelle:

\begin{table}[!h]
    \centering
    \caption{}
    \begin{tabular}{c c c c c c}
      \toprule
      $C_k\,/\,nF$ & $\eta_{exp}$ & $f_+\,/\,kHz$ & $f_-\,/\,kHz$ & $\eta_{th}$ & $\text{Abweichung}\,/\,\text{\%}$ \\
      \midrule
      9,99 & 12 & 29,694 &  & & \\
      8,00 & 11 & 29,694 & & & \\
      6,47 & 9  & 29,694 & & & \\
      5,02 & 7  & 29,694 & & & \\
      4,00 & 6  & 29,694 & & & \\
      3,00 & 5  & 29,694 & & & \\
      2,03 & 4  & 29,694 & & & \\
      1,01 & 2  & 29,694 & & & \\
      \bottomrule
    \end{tabular}
\end{table}

\subsection{Bestimmen der Eigenfrequenzen}

Die theoretisch und experimentell bestimmten Eigenfrequenzen($f_-^{th}, f_+^{th}, f_-^{exp} \text{und}\,f_+^{exp}$) sind, in abhängigkeit zum
Koppelkondensator $C_k$, in Tabelle (\ref{tabeigenfr}) aufgelistet. Dazu kommen noch die Abweichungen zwischen den theoretischen und experimentellen
Werten:

\begin{table}[!h]
    \centering
    \caption{}
    \label{tabeigenfr}
    \begin{tabular}{c c c c c c c}
      \toprule
      $C_k\,/\,nF$ & ${f_+}^{exp}\,/\,kHz$ & ${f_-}^{exp}\,/\,kHz$ & ${f_+}^{th}\,/\,kHz$ & ${f_-}^{th}\,/\,kHz$ & $\text{Abw.}f_+\,/\,\text{\%}$ & $\text{Abw.}f_-\,/\,\text{\%}$ \\
      \midrule
      9,99 & 30,3 & 33,56 & 29,694 & & 2.041 & \\
      8,00 & 30,3 & 34,01 & 29,694 & & 2.041 & \\
      6,47 & 30,3 & 34,72 & 29,694 & & 2.041 & \\
      5,02 & 30,3 & 35,21 & 29,694 & & 2.041 & \\
      4,00 & 30,3 & 36,50 & 29,694 & & 2.041 & \\
      3,00 & 30,3 & 37,59 & 29,694 & & 2.041 & \\
      2,03 & 30,3 & 40,32 & 29,694 & & 2.041 & \\
      1,01 & 30,3 & 47,17 & 29,694 & & 2.041 & \\
      \bottomrule
    \end{tabular}
\end{table}
